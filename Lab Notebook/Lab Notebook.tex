\documentclass[12pt,twoside]{article}

\usepackage{amsmath}
\usepackage{fancyhdr}
\usepackage{graphicx}
\usepackage{amssymb}
\usepackage{wrapfig}
\usepackage[colorlinks=true, linkcolor=red, urlcolor=blue, citecolor=blue]{hyperref}

\newenvironment{tight_enumerate}{
\begin{enumerate}
  \setlength{\itemsep}{0pt}
  \setlength{\parskip}{0pt}
}{\end{enumerate}}

\newenvironment{tight_itemize}{
\begin{itemize}
  \setlength{\itemsep}{0pt}
  \setlength{\parskip}{0pt}
}{\end{itemize}}

\setlength{\oddsidemargin}{0pt}
\setlength{\evensidemargin}{0pt}
\setlength{\textwidth}{6.5in}
\setlength{\topmargin}{0in}
\setlength{\textheight}{8.5in}
\setlength{\headheight}{15pt}

\newcommand{\andrew}{Andrew Xia}
\newcommand{\name}{6.115 Final Project Lab Notebook}
\renewcommand{\thesubsection}{\thesection.\alph{subsection}}
\renewcommand{\sectionmark}[1]{\markboth{#1}{}} % set the \leftmark


\pagestyle{fancy}
\fancyhead[R]{\andrew}
\fancyhead[L]{\leftmark} % 1. sectionname
\fancyhead[C]{\name}


\begin{document} 
\begin{center} {\bf \Large \name}
\\ \emph{3D Tic Tac Toe}
\\ {\bf \andrew}
\end{center}
\section{Overview}
{\em April 20, 2016 Proposal:} For my final project, I hope to build a 3D touchless tracking interface and connect it with my 8051 and PSoC to allow users to play 3D tic tac toe with either each other or a computer AI. 
\\
\\ \emph{What interested you in the idea?} I am interested in this idea because I want to work on a project that has both an interesting and flashy hardware and software component. Writing the AI software for my 3D tic tac toe board will be a very rewarding and educational experience for me, and building the 3D tracing interface will also be very fun. 
\\ 
\\ \emph{Why is this project interesting?} This project is interesting because while 2D tic tac toe is a game familiar to all, 3D tic tac toe represents a game that while it is familiar, it can also be novel and entertaining. This project also uses a 3D touchless interface for user input, which is another novel way of interacting with software, as we are used to touching buttons or screens to communicate with devices in our daily lives. 

\subsection{Hardware Overview}
The key hardware component of my project will be building a 3D touchless interface, which is shown in the image to the side. The three aluminum foils (or a similar material) on the three sides are the capacitors that the human finger complements, and thie distance from each of the three capacitors will definite the capacitance of the capacitance, which we can measure using an ADC, and effectively pinpoint the 3D location of the human hand. 
\\ I will use a 74LS245 chip to connect my PSoC to a VGA cable which can then display information on a screen. I will use what was covered in lecture to complete this task. 
\\ Below is a preliminary hardware schematic of my final project:
\begin{center}\includegraphics[width = 120mm]{../Proposals/Hardware_final.png} \end{center}

\subsection{Software Overview}
As for my software, I will code my project in C, using the PSoC as my main control module, as it would would be more efficient to do so over writing code in assembly for the 8051. The Tic Tac Toe AI is not too software heavy, so I hope to develop the basic framework for my code and spend the rest of my extra time focusing on hardware improvements. 
\\ Here is the high level software workflow for my final project:
\begin{center}\fbox{\begin{minipage}{35em}

\begin{tight_enumerate}
\item Initiate \& Reset Tic Tac Toe Board
\item Mode 1: two player Tic Tac Toe
	\begin{tight_enumerate}
	\item User can view \& rotate, using 74L922 or capacitive-based sensors, current 3D tic tac toe board on VGA display
	\item User adds piece to the board by leaving hand in position of 3D touchless tracking interface for extended amound of time
	\item After entering piece on board, the other player can play. 
	\end{tight_enumerate}
\item Mode 2: one player AI Tic Tac Toe
	\begin{tight_enumerate}
	\item User can view \& rotate, using 74L922 or capacitive-based sensors, current 3D tic tac toe board on VGA display
	\item User adds piece to the board by leaving hand in position of 3D touchless tracking interface for extended amound of time
	\item After entering piece on board, computer plays against user and submits his move through AI algorithm. 
	\end{tight_enumerate}
\item Reset the game and enter either mode through the 74C922 Keypad
\end{tight_enumerate}

\end{minipage}} \end{center}

\subsection{Format of my Lab Notebook}
I plan on formatting my lab notebook in this latex file, in which each day I work on the lab will be a section, and I can also reorganize my lab notebook to be thematically organized by the module that I will work on. 
\\ Each day, or section, will contain an {\em introduction} section in which I discuss the goals that I have for the day, and I will detail the tasks that I will work on in the subsections for each day. 

%%%%%%%%%%%%%%%%%%%%%%%%%%%%%%%%%%%%%%%%%%

\newpage
\section{April 21, 2016}
Goals I have for today / {\bf done} if task completed
\begin{tight_itemize}
\item Set up github for version control of my lab notebook and code {\bf done}
\item Build my initial capacitance based sensor
\item Set up my computer so I can code in C {\bf done}
\end{tight_itemize}

\subsection{Setting up github}
I am following the tutorial \href{https://help.github.com/articles/adding-an-existing-project-to-github-using-the-command-line/}{here} to create a new github repository. It will be located in my 6.115 dropbox folder, and on my github account under the repository \href{https://github.com/qandrew/6.115-final-project}{6.115-final-project}

\subsection{Building my initial capacitance based sensor}
References include this \href{http://makezine.com/projects/a-touchless-3d-tracking-interface/}{Makezine link}. 
\\ Notes on building a capacitance based sensor:
\begin{tight_enumerate}%[itemsep=5mm]
\item Beacuse humans have very little capacitance $~ 10 pF$, the time constant to which the plate would charge up based on the distance of the finger to the plate should be very small. Thus, I may need to think of more intelligent methods to build an effective capacitance based sensor. 
\item I can design different circuits
\end{tight_enumerate}

\subsection{Software development}
To install {\bf C}, I am following this \href{https://en.wikibooks.org/wiki/C_Programming/What_you_need_before_you_can_learn}{website} to install a C compiler for my computer. It is in fact better to code in Linux rather than $C$, so I will be using linux/ubuntu when needing to test my Tic Tac Toe code on my computer.
\\
\\ To install the Cypress PSoC creator IDE, I am following the instructions \href{http://www.cypress.com/products/psoc-software}{here}. Having PSoC Creator on my computer should help increase my productivity and workflow. 
\\
\\ I completed a two player Tic Tac Toe game coded using python today. The logic of the game is documented in my code. I plan on investigating more into AI algorithms, potentially using minimax or alpha beta pruning. \href{http://www.half-real.net/tictactoe/}{This website} has pretty good information on evaluation methods and algorithms for 3D Tic Tac Toe. 
\begin{center}\includegraphics[width = 100mm]{Pics/4-21python.png} \end{center}
Above is an image of my python 3D tic tac toe script running

%%%%%%%%%%%%%%%%%%%%%%%%%%%%%%%%%%%%%%%%%%

\section{April 24 2016}

Goals I have for today / {\bf done} if task completed
\begin{tight_itemize}
\item Learn how to use Cypress EDS Laser Cutter  {\bf done}
\item Look into VGA connection to PSoC, how to connect my capacitance based sensor
\end{tight_itemize}
\begin{wrapfigure}{R}{0.3\textwidth}
\centering
\includegraphics[width = 30mm]{Pics/4-24b.jpg} 
\caption{\label{fig:frog1}test.} 
\end{wrapfigure}
\subsection{Using the Cypress EDS Laser Cutter}
I printed out a circular acryllic piece today with ny name on it. Some things to remember: I should use CorelDRAW or solid works to design my cuts. The {\em red} line must be set to hairline width for cuts. The {\em blue} line must be set to hairline width for scones. The {\em black} areas must can be set to anything for rasters. 

\subsection{Working on my capacitance based sensor}
The initial circuit that I tested did not work. It was picking up a lot of 60Hz noise (which is coming from the AC power that exists all around the lab). The top resistor has value $220k\Omega$ and the bottom resistor has value $10k\Omega$. 
%\begin{wrapfigure}{R}{0.3\textwidth}
%\centering
\begin{center} \includegraphics[width = 50mm]{Pics/4-24a.jpg}  \end{center} %\caption{\label{fig:frog1}test.} 
%\end{wrapfigure}
This circuit does not work.
\\ 
\\ I am now trying a new circuit, designed as follows according to the Makezine article
\begin{center} \includegraphics[width = 50mm]{Pics/4-24c.jpg}  \end{center}
I am able to sense when I touch the cardboard aluminum foil capacitor plate, but I am unable to get a different capacitor RC time constant reading when I vary my hand distance from the aluminum foil. 
\\ I can consider designing a \href{http://www.radio-electronics.com/info/circuits/rc_notch_filter/twin_t_notch_filter.php}{twin t notch filter} to filter out the 60Hz noise. 


%%%%%%%%%%%%%%%%%%%%%%%%%%%%%%%%%%%%%%%%%%

\section{April 25 2016}

Goals I have for today / {\bf done} if task completed
\begin{tight_itemize}
\item Finalize design for my user input % {\bf done}
\item Look into connecting PSoC to 8051 and VGA display
\item Continue Building my 3D Tic Tac Toe AI {\bf [done]}
\end{tight_itemize}

\subsection{Designing the Capacitance Based Sensor } 
This is the twin t notch filter that I have in mind: 
\begin{center} \includegraphics[width = 50mm]{Pics/4-25a.jpg}  \end{center}
The formula for the cutoff frequency is $f_c = \frac{1}{2\pi RC}$, and I am using $C = 100pF$ and $R = 27k\Omega$. However, I am unable to filter out a default 60Hz. 

\subsection{Connecting PSoC to 8051}
Check out \href{http://web.mit.edu/6.115/www/document/cable_blackbird.pdf}{this link} to see how to connect my 8051 to my PSoC. 
\\ Today, I soldered the Sparkfun connector piece, and I got the LA checkoff showing that I could successfully power my PSoC using R31JP power. I finished the pdf linked above and got the following to display:
\\ (insert image)

\subsection{Using the 74LS245}
The purpose of the 74LS245 chip is to connect my PSoC to a VGA display, so I can effectively visualize the 3D tic tac toe game. 

\subsection{Building my 3D Tic Tac Toe AI}
\begin{wrapfigure}{R}{0.42\textwidth}
\centering
\includegraphics[width = 0.4\textwidth]{Pics/4-25b.png} 
\end{wrapfigure}
I completed building my AI for my 3D Tic Tac Toe Game in python today. I thought about running a brute force tree search to exhaust all combinations of the 3D Tic Tac Toe Game, but that would be on the order of 64! moves and would be a bit complicated. Instead, I wrote an evaluation metric to evaluate the values of certain locations on the boards, and the AI would place its next move based on the highest scoring location. My scoring works as follows, for a certain row/diagonal of 4 that does not have both an AI and player move:
\begin{tight_enumerate}
\item If the AI has a 3 in a row, connect it
\item If the opponent has a 3 in a row, block it
\item If the opponent has a 2 in a row, block it
\item If the AI has a 2 in a row, connect it
\item If the AI has a 1 in a row, connect it
\item If the row has nothing, add a move
\item If the opponent has a 1 in a row, block it
\end{tight_enumerate}
The cumulative sum of the scores after checking all potential connect 4 paths will determine the next place to play. You can take a look at the code \href{https://github.com/qandrew/6.115-final-project/blob/master/Code/tic_tac_toe_ai.py}{here}.
\\
\\ As for further work to be done, I should (a) port the game to C to put on the PSoC, (b) consider difficulty settings for the game, (c) WORK ON THE HARDWARE ASPECT!

%%%%%%%%%%%%%%%%%%%%%%%%%%%%%%%%%%%%%%%%%%

\newpage
\section{April 26 2016}

Goals I have for today / {\bf done} if task completed
\begin{tight_itemize}
\item Connect PSoC to VGA display % {\bf done}
\item Code 3D Tic Tac Toe onto PSoC
\item Buy capacitance based sensing chips
\end{tight_itemize}

\subsection{Connect PSoC to VGA display}
I got a PSoC module from Prof. Leeb today that will allow me to connect my PSoC to the module using a Serial Port, and the module will convert the serial port data to something displayable on the monitor display through a VGA display port. 
\\ I am missing a male to male serial port connector. 

%%%%%%%%%%%%%%%%%%%%%%%%%%%%%%%%%%%%%%%%%%
%TEMPLATE%

\newpage
\section{DATE 2016}

Goals I have for today / {\bf done} if task completed
\begin{tight_itemize}
\item Task % {\bf done}
\item Task 
\end{tight_itemize}


\end{document}